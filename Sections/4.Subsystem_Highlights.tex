%\textit{Subsystem Highlights: 3-5 pages for this one too.  He suggests placing the Subsystem Decomp in this section and using CAD models and schematics to illustrate how they combine to create our entire system.  Here is where we will directly compare the final subsystem choices with those we didn’t go with. Again, think of the “crucial” choices and how some of the failed or unchosen designs led to our final choices. Don’t go crazy with figures and pictures though.  Refer to the Appendix for extra details and images that don’t quite belong in the body of the report.}
 
\subsection{Mating Rings}

%(Main Point): Aluminum 6061-T6 Mating rings incorporated design shown below (ref figure), in order to adhere and connect all subsystem components together.  Two different designs were incorporated for each associated liner design (ref figure), with different lapping portion dimensions etc (reiterate, ref figures)

%(Discuss various designs, need to figure out how we want to discuss each design, do we do them all in one slew or do we basically do 2 or 3 iterations of all these sections so that all the info for each design is in one place?  Show CAD, etc..)
Using Aluminum 6061-T6, the mating rings were designed in such a way to integrate all subsystem components together. Aluminum 6061-T6 is a common available material that has a high strength-to-weight ratio relative to other metals and no ductile to brittle transition temperature. Choosing an alternative material such as steel, would introduce galvanic coupling when mated with other aluminum parts and introduce corrosion issues. The mating ring design was accomplished using a cylindrical shape with two open ends to allow compression molding of the composite material.\textbf{(ref figure/CAD)} A thin feature at the bottom of the ring provides a surface for the composite sandwich structure to adhere on the outside, and an offset on the interior to seat the PTFE liner(ref fig). Twelve holes (dims?) were tapped at the top surface of the rings in order to fasten the tank’s end caps to the associated mating rings. The lip of the ring (ref fig) was designed as a placeholder which will be used as a mating feature to integrate the tank to the rocket’s airframe. With the knowledge that alignment during the shrink fitting process would be important, a triangular 'keyway' was incorporated against the lip of the mating ring in between the tapped holes (ref fig).  This keyway provides a method of straightening the end cap during the shrink fit application so that all fastener holes are aligned once completed.

\subsection{End Caps}

%(Main Point):  Aluminum 6061-T6 End caps were incorporated using two different designs shown below (ref fig) due to their respective PTFE liner designs.  To be used with the sheet, a flat end cap secured into place with hex fasteners and a PTFE gasket was applied (ref fig).  With the tube design, a larger, shrink fitted end cap was applied (ref fig), to provide a compression seal on the PTFE tube liner.

Also designed using Aluminum 6061-T6, the end caps incorporated a 1/16" thickness plate with a rounded central portion to account for the addition of fitting valves.  Twelve through holes were placed along the outer rim of each end cap, allowing fasteners to pass through and secure into the associated mating rings.  In between through hole locations, a triangular cut was added in order to align with the keyway mentioned above, to aid in the shrink fitting process during tank fabrication.  On the bottom portion of each cap, a longitudinal surface was recessed down into the tank, in order to provide a seal by compressing the PTFE liner in between this portion of the end cap against the mating ring.

\subsection{Sandwich Layering (Carbon Fiber, Nomex, Adhesive)}

%(Main Point):  Carbon Fiber & Nomex honeycomb sandwich layering designs were incorporated based on the development and materials available from the 2014 LV3.0 Composite Airframe Capstone Team.  Cytec Metlbond Adhesive was used to secure all material layers in place.  (ref figs).

Making the tank out of composite material is especially desirable in an aerospace design due to their outstanding strength-to-weight ratio, which are superior to metals. The carbon fiber sandwich was designed to provide the tank with structural support from external loads including the rocket’s weight and thrust, and internal loads from the pressure of the oxidizer. 
The sandwich consist of a honeycomb core material with carbon fiber face sheets on either side. Structural adhesive is placed between each of the layer to adhere them together. The carbon fiber face sheets are used to provide strength from hoop and axial loads on the tank. Due to carbon fibers anisotropic nature, its strength against traverse loading is a fraction of its strength in the hoop and axial direction \textbf{(Cite ISBN:978-0-471-73696-7)}. To strengthen the composite from traverse loads, Nomex honeycomb is sandwiched between the carbon fiber face sheets. 


\subsection{Liner Materials}

%(Main Point):To prevent LOX from leaking into  the prepreg carbon fiber layers, PTFE was elected as the preferred tank liner material due to its wide operating temperature range, excellent chemical resistance (specifically with LOX), and low permeability. 

 The fluorocarbon family is particularly well suited for low temperature applications because they are the only known materials that retain a measurable ductility at temperature very close to absolute zero (-269\celsius). This makes them suitable for use as insulators and as static seals at these temperatures \textbf{(CITE: PLASTICSLOWTEMP.PDF)}.
 
 This liner would have to survive the carbon fiber curing process or be integrated after laying up the structural shell of the tank. It would also need to be held in place by either an adhesive or some mechanical joining method. The team was provided with a structural adhesive that could be cured in tandem with the carbon fiber, so it made sense to place the liner and cure simultaneously. The decision to co-cure required that the liner material maintain integrity over the full range of temperatures between the boiling point of liquid oxygen and the curing temperature of the composite materials. Only two materials were found with these capabilities. They were ethylene tetrafluoroethylene (ETFE) and polytetrafluoroethylene (PTFE). PTFE was chosen on the basis of availability and cost. 
     
Two different PTFE designs were incorporated, one using an etched sheet, the other using a ⅛”-1/16” wall thickness tube machined in-house.

As previously discussed, the major challenge of this project has been a way to safely isolate LOX fuel from the chemically incompatible carbon fiber epoxy.  The primary material chosen to act as this liner was PTFE (likely should denote full non-abbreviated name somewhere up above in a beginning section).  There are a very limited number of (Discuss permeability, show CAD, discuss chemical inertness)


\subsection{Sealant/Fittings/Fasteners etc?}

%(Main Point): (Insert proper name)  Cry sealant ‘caulking’ was applied to all interior edges of the tank where leak paths may occur.  This sealant is specified to operate at cryogenic temperatures to provide a secure seal to the interior of the tank.  PTFE gaskets were also secured & compressed in between the end cap & mating rings (sheet design) to provide a seal.

In order to provide a seal to the tank between the cap and ring interfaces, gaskets were designed and cut from PTFE Gasket material (is there a proper name?) in order to provide a secure cryogenic seal at the given temperatures/pressures.

\newpage